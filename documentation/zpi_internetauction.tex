\documentclass[a4paper]{article}
\usepackage{polski}
\usepackage[cp1250]{inputenc}
\usepackage{url}
\usepackage{hyperref}

\title{\vspace{3cm} \bf{\Huge Internetowy\\ \vspace{0.2cm} Serwis\\ \vspace{0.4cm} Aukcyjny}}
\author{{\em Jakub Niewczas, Damian Klimek, Tomasz Zabrzewski, Łukasz Myszkowski}}
\date{}

\begin{document}
	
	\begin{titlepage}
		\thispagestyle{empty}
		\maketitle
		\vspace{8cm}
		\begin{center}
			Zespołowe przedsięwzięcie inżynierskie\\[2mm]
			
			Informatyka\\[2mm]
			
			Rok. akad. 2017/2018, sem. I\\[2mm]
			
			Prowadzący: dr hab. Marcin Mazur
		\end{center}
	\end{titlepage}
	
	\tableofcontents
	\thispagestyle{empty}
	
	\newpage
	
	\section{Opis projektu}
	
	\subsection{Członkowie zespołu}
	
	\begin{enumerate}
		\item Damian Klimek (kierownik projektu);
		\item Jakub Niewczas;
		\item Tomasz Zabrzewski;
		\item Łukasz Myszkowski;
	\end{enumerate}
	
	\subsection{Cel projektu (produkt)}
	
	Głównym i najważniejszym celem projektu jest stworzenie platformy webowej, która pojęcie ,,zakupy'' pchnie krok dalej. Potencjalni klienci będą mieć możliwość robienia zakupów \emph{(tj. sprzedawanie i kupowanie przedmiotów)} bez wychodzenia z domu. Kolejną ważną rzeczą jest zaimplementowanie modułu indywidualnych kont dla użytkowników \emph{(logowanie się na swoje konto oraz rejestrowanie nowego)}. Również stawiamy nacisk na stworzenie miłego dla oka, prostego a przede wszystkim intuicyjnego designu strony internetowej aby użytkownicy mogli w łatwy sposób przemieszczać się po platformie i szybko wyszukiwać interesujące ich przedmioty.
	
	\subsection{Potencjalny odbiorca produktu (klient)}
	
	Konkretnym klientem może być każdy, kto potrzebuje pieniędzy przez co dostaje możliwość sprzedania czegokolwiek lub osoby potrzebujące jakiegoś dobra. Widełki wieku klientów nie są definiowalne - każdy kto posiada dostęp do internetu może skorzystać z usług jakie oferuje produkt.
	
	\subsection{Metodyka}
	
	Projekt będzie realizowany przy użyciu (zaadaptowanej do istniejących warunków) metodyki {\em Scrum}. 
	
	\section{Wymagania użytkownika}
	
	\subsection{User story 1}
	Jako nowy użytkownik serwisu chciałbym mieć możliwość założenia swojego indywidualnego konta, żebym mógł korzystać w pełni z usług dostarczanych przez serwis.
	
	\subsection{User story 2}
	Jako użytkownik chciałbym mieć możliwość zalogować się na swoje indywidualne konto, by móc korzystać z wszystkich usług platformy.
	
	\subsection{User story 3}
	Jako użytkownik chciałbym mieć możliwość wystawienia przedmiotu na aukcję, żebym mógł sprzedawać przedmioty oraz zarabiać pieniądze.
	
	\subsection{User story 4}
	Jako użytkownik podczas wystawiania przedmiotu na aukcję chciałbym mieć możliwość wybrania formy aukcji \emph{(tj. licytacja, opcja ,,KUP TERAZ'' lub obie formy)}, żebym mógł sprzedać przedmiot w wybranej formie aukcji.
	
	\subsection{User story 5}
	Jako użytkownik podczas wystawiania przedmiotu na aukcję chciałbym mieć możliwość wprowadzenia opisu aukcji:
	\begin{itemize}
		\item konkretna nazwa aukcji;
		\item dodatkowe informację o sprzedawanym przedmiocie;
		\item cena \emph{(wywoławcza podczas licytacji lub stała za przedmiot podczas opcji ,,KUP TERAZ'')};
		\item wgranie zdjęcia przedmiotu;
	\end{itemize}
	żebym mógł zachęcić potencjalnego kupca do zakupu przedmiotu.
	
	\subsection{User story 6 (Opcjonalnie)}
	Jako użytkownik chciałbym mieć możliwość komentowania produktów innych osób wystawiających przedmioty na aukcji, , żebym mógł dowiedzieć się interesujących informacji od innych osób, które wcześniej kupiły przedmiot na jego temat bądź podzielić się własną opinią. 
	
	\subsection{User story 7}
	Jako użytkownik chciałbym mieć możliwość dodania przedmiotów które mnie interesują do zakładek \emph{ (tj. Ulubione)}, żebym mógł szybko i łatwo je odnaleźć. 
	
	\subsection{User story 8}
	Jako użytkownik chciałbym mieć możliwość dodania swoich danych:
	\begin{itemize}
		\item imię;
		\item nazwisko;
		\item miejscowość;
		\item numer telefonu;
		\item e-mail;
		\item numer konta bankowego;
	\end{itemize}
	żeby umożliwić innym użytkownikom kontakt ze mną i rozliczeń.
	
	\subsection{User story 9}
	Jako użytkownik chciałbym mieć możliwość zmiany:
	\begin{itemize}
		\item istniejącego  hasła na nowe;
		\item istniejącego e-maila na nowy;
		\item danych osobowe na nowe;
	\end{itemize}
	żeby zaktualizować email oraz wprowadzić nowe hasło bądź też poprawić błąd w danych osobowych lub całkowicie je zmienić.
	
	\subsection{User story 10}
	Jako użytkownik chciałbym mieć możliwość zobaczenia ile razy odwiedzono moją aukcję \emph{ (tj. licznik odsłon)}, żebym mógł zobaczyć jakie jest zainteresowanie moją aukcją.
	
	\subsection{User story 11}
	Jako użytkownik chciałbym mieć możliwość edytowania danych aukcji \emph{ (tj. cena, opis)} oraz usunięcia danej aukcji, żebym mógł poprawić błedy lub zaktualizować dane aukcji.
	
	\subsection{User story 12}
	Jako użytkownik chciałbym mieć możliwość wyszukiwania interesujących mnie aukcji poprzez podanie słów kluczowych w wyszukiwarce, żebym mógł znaleźć przedmiot, który chciałbym zakupić.
	
	\subsection{User story 13}
	Jako użytkownik chciałbym by po dokonaniu transakcji (kup teraz bądź wygrana licytacja) wysyłała się do kupującego wiadomość z moim numerem konta oraz informacjami dotyczącymi transakcji, by mógł on dokonać wpłaty za zakupiony przedmiot.
	
	\subsection{User story 14}
	Jako użytkownik chciałbym mieć możliwość sprawdzenia starych aukcji które zostały zakończone (przedmiot sprzedano, przedmiot kupiono) bądź też wygasły, żebym mógl sprawdzić historie swoich transakcji.
	 
	\subsection{User story 15}
	Jako użytkownik chciałbym mieć możliwość wysłania wiadomości prywatnej do innego użytkownika serwisu, żebym mógł się z nim skontaktować. 
	
	\subsection{User story 16}
	Jako użytkownik chciałbym mieć możliwość przeglądania wszystkich dostępnych aukcji na serwisie w postaci listy wraz z ich podstawowymi danymi \emph{(tj. nazwa aukcji, forma aukcji, cena, data rozpoczęcia oraz zakończenia aukcji)}, żebym mógł przeglądać aukcje oferowane przez serwis.
	
	\subsection{User story 17}
	Jako użytkownik chciałbym mieć możliwość zakupu produktu natychmiast (opcja kup teraz), bądź licytowania go (opcja licytuj), żebym mógł go kupić.
		
	\section{Harmonogram}
	
	\subsection{Rejestr zadań (Product Backlog)}
	
		\begin{itemize}
		\item Data rozpoczęcia: 24.10.2017.
		\item  Data zakończenia: 14.11.2017.
	\end{itemize}
	
	\subsection{Sprint 1}
	
	\begin{itemize}
		\item Data rozpoczęcia: \emph{31.10.2017};
		\item Data zakończenia: \emph{14.11.2017};
		\item Scrum Master: \emph{Łukasz Myszkowski};
		\item Product Owner: \emph{Tomasz Zabrzewski};
		\item Development Team: \emph{Jakub Niewczas, Damian Klimek}
	\end{itemize}
	
	\subsection{Sprint 2}
	
	\begin{itemize}
		\item Data rozpoczęcia: 14.11.2017.
		\item  Data zakończenia: 28.11.2017.
		\item Scrum Master: Tomasz Zabrzewski.
		\item Product Owner: Damian Klimek.
		\item Development Team: Łukasz Myszkowski, Jakub Niewczas.
	\end{itemize}

	\subsection{Sprint 3}

	\begin{itemize}
		\item Data rozpoczęcia: 28.11.2017.
		\item  Data zakończenia: 12.12.2017.
		\item Scrum Master: Damian Klimek.
		\item Product Owner: Jakub Niewczas.
		\item Development Team: Tomasz Zabrzewski, Łukasz Myszkowski.
	\end{itemize}
	
	\subsection{Sprint 4}

	\begin{itemize}
		\item Data rozpoczęcia: 12.12.2017.
		\item  Data zakończenia: 09.01.2018.
		\item Scrum Master: Jakub Niewczas.
		\item Product Owner: Damian Klimek.
		\item Development Team: Tomasz Zabrzewski, Łukasz Myszkowski.
	\end{itemize}

	\subsection{Sprint 5}

	\begin{itemize}
		\item Data rozpoczęcia: 09.01.2018.
		\item  Data zakończenia: 23.01.2018.
		\item Scrum Master: Łukasz Myszkowski.
		\item Product Owner: Tomasz Zabrzewski.
		\item Development Team: Damian Klimek, Jakub Niewczas.
	\end{itemize}
	
	\section{Product Backlog}
	
	\subsection{Backlog Item 1}
	\paragraph{Tytuł zadania:} Baza danych
	\paragraph{Opis zadania:} Przygotowanie struktury bazy danych.
	\paragraph{Priorytet:} Wysoki
	\paragraph{Definition of Done:} Baza danych powinna zawierać tyle tabel ile potrzeba na stworzenie aplikacji. Środowisko bazodanowe \textbf{MySQL}. Baza danych serwisu musi posiadać następujące struktury.
	\vspace{0.5cm}
	\\Struktura tabeli \textbf{users}:
	\begin{itemize}
		\item \textbf{id} $\to$ [typ - \verb|INT|, \verb|AUTO_INCREMENT|];
		\item \textbf{username} $\to$ [typ - \verb|VARCHAR(32)|, kodowanie - \verb|UTF8_GENERAL_CI|];
		\item \textbf{password} $\to$ [typ - \verb|VARCHAR(32)|, kodowanie - \verb|UTF8_GENERAL_CI|];
		\item \textbf{firstname} $\to$ [typ - \verb|VARCHAR(64)|, kodowanie - \verb|UTF8_GENERAL_CI|];
		\item \textbf{surname} $\to$ [typ - \verb|VARCHAR(64)|, kodowanie - \verb|UTF8_GENERAL_CI|];
		\item \textbf{email} $\to$ [typ - \verb|VARCHAR(64)|, kodowanie - \verb|UTF8_GENERAL_CI|];
		\item \textbf{phone} $\to$ [typ - \verb|VARCHAR(9)|, kodowanie - \verb|UTF8_GENERAL_CI|];
		\item \textbf{place} $\to$ [typ - \verb|VARCHAR(128)|, kodowanie - \verb|UTF8_GENERAL_CI|];
		\item \textbf{bank} $\to$ [typ - \verb|VARCHAR(26)|, kodowanie - \verb|UTF8_GENERAL_CI|];
	\end{itemize}

	\vspace{0.5cm}
	Struktura tabeli \textbf{categories}:
	\begin{itemize}
		\item \textbf{id} $\to$ [typ - \verb|INT|, \verb|AUTO_INCREMENT|];
		\item \textbf{name} $\to$ [typ - \verb|VARCHAR(64)|, kodowanie - \verb|UTF8_GENERAL_CI|];
	\end{itemize}

	\vspace{0.5cm}
	Struktura tabeli \textbf{auctions}:
	\begin{itemize}
		\item \textbf{id} $\to$ [typ - \verb|INT|, \verb|AUTO_INCREMENT|];
		\item \textbf{name} $\to$ [typ - \verb|VARCHAR(128)|, kodowanie - \verb|UTF8_GENERAL_CI|];
		\item \textbf{category} $\to$ [typ - \verb|INT|, relacja z kolumną id z tabeli categories];
		\item \textbf{buy\_prize} $\to$ [typ - \verb|VARCHAR(16)|, kodowanie - \verb|UTF8_GENERAL_CI|];
		\item \textbf{bidding\_prize} $\to$ [typ - \verb|VARCHAR(16)|, kodowanie - \verb|UTF8_GENERAL_CI|];
		\item \textbf{description} $\to$ [typ - \verb|VARCHAR(512)|, kodowanie - \verb|UTF8_GENERAL_CI|];
		\item \textbf{image} $\to$ [typ - \verb|VARCHAR(256)|, kodowanie - \verb|UTF8_GENERAL_CI|];
		\item \textbf{last\_bidder} $\to$ [typ - \verb|INT|, relacja z kolumną id z tabeli users];
	\end{itemize}

	\vspace{0.5cm}
	Struktura tabeli \textbf{comments}:
	\begin{itemize}
		\item \textbf{id} $\to$ [typ - \verb|INT|, \verb|AUTO_INCREMENT|];
		\item \textbf{message} $\to$ [typ - \verb|VARCHAR(512)|, kodowanie - \verb|UTF8_GENERAL_CI|];
		\item \textbf{owner} $\to$ [typ - \verb|INT|, relacja z kolumną id z tabeli users];
		\item \textbf{auction} $\to$ [typ - \verb|INT|, relacja z kolumną id z tabeli auctions];
	\end{itemize}

	\vspace{0.5cm}
	Struktura tabeli \textbf{messages}:
	\begin{itemize}
		\item \textbf{id} $\to$ [typ - \verb|INT|, \verb|AUTO_INCREMENT|];
		\item \textbf{owner} $\to$ [typ - \verb|INT|, relacja z kolumną id z tabeli users];
		\item \textbf{recipient} $\to$ [typ - \verb|INT|, relacja z kolumną id z tabeli users];
		\item \textbf{title} $\to$ [typ - \verb|VARCHAR(128)|, kodowanie - \verb|UTF8_GENERAL_CI|];
		\item \textbf{message} $\to$ [typ - \verb|VARCHAR(512)|, kodowanie - \verb|UTF8_GENERAL_CI|];
		\item \textbf{readed} $\to$ [typ - \verb|INT|];
	\end{itemize} 
	
	\subsection{Backlog Item 2}
	\paragraph{Tytuł zadania:} Rejestracja
	\paragraph{Opis zadania:} Możliwość założenia nowego konta;
	\paragraph{Priorytet:} Wysoki
	\paragraph{Definition of Done:} Każda osoba powinna mieć możliwość założenia swojego indywidualnego konta. Należy stworzyć formularz z podstawowymi danymi (\emph{tj. login, hasło, imię, nazwisko, e-mail, miejscowość, numer telefonu, numer konta bankowego}). Wpisane przez użytkownika dane muszą być pobierane metodą \emph{POST}. Finalnym efektem rejestracji jest wysłanie zapytania do bazy danych, który utworzy nowy rekord w tabeli \emph{users} z danymi użytkownika.
	
	\subsection{Backlog Item 3}
	\paragraph{Tytuł zadania:} Zabezpieczenie hasła
	\paragraph{Opis zadania:} Wymuszenie od użytkownika wprowadzenia danych spełniających określone kryteria;
	\paragraph{Priorytet:} Niski
	\paragraph{Definition of Done:} Użytkownik tworzący nowe konto musi podać hasło składające się z minimum 8 znaków w tym:
	\begin{itemize}
		\item wielkie litery od A do Z;
		\item małe litery od a do z;
		\item cyfry od 0 do 9;
		\item znaki niealfabetyczne (np. \verb|!, @, #, &|);
	\end{itemize}
	Należy stworzyć skrypt używający \emph{wyrażenia regularne}, które w łatwy sposób mogą sprawdzić poprawność hasła z wyżej ustalonymi zasadami. Hasło, które będzie podawał użytkownik rejestrując lub logując się na swoje konto powinno być niewidoczne dla osób postronnych (wykropkowane).

	\subsection{Backlog Item 4}
	\paragraph{Tytuł zadania:} Weryfikowanie danych
	\paragraph{Opis zadania:} Sprawdzanie wprowadzanych danych przez użytkownika podczas rejestracji;
	\paragraph{Priorytet:} Wysoki
	\paragraph{Definition of Done:} Każda osoba wprowadzająca dane do serwisu podczas rejestracji powinna podać poprawne dane:
	\begin{itemize}
		\item imię - powinno zawierać od 3 do 32 znaków, w tym tylko od a do z;
		\item nazwisko - powinno zawierać od 3 do 32 znaków, w tym tylko od a do z;
		\item e-mail - powinien przypominać maskę (tj. mójemail09@domena.pl);
		\item miejscowość - powinna zawierać od 3 do 64 znaków, w tym tylko od a do z;
		\item numer telefonu - powinna składać się z 9 cyfr.
		\item konto bankowe - powinno składać się z 26 cyfr.	
	\end{itemize}
	Gdy użytkownik nie zastosuje się do wyżej wymienionych zasad, należy wysłać do niego w postaci \emph{JavaScript Popup Alert box}, wiadomość zwrotną podającą, w którym miejscu popełnił błąd oraz jak poprawnie wypełnić input formularza.
	
	\subsection{Backlog Item 5}
	\paragraph{Tytuł zadania:} Logowanie
	\paragraph{Opis zadania:} Możliwość zalogowania się na swoje konto;
	\paragraph{Priorytet:} Wysoki
	\paragraph{Definition of Done:} Stworzenie formularza logowania (tj. login oraz hasło) pobierające owe dane, następnie w postaci zapytania MySQL musi sprawdzać czy istnieją dane w bazie danych oraz czy pasują do jednego użytkownika. Należy zaimplementować wiadomość zwrotną gdy użytkownik wpisał złe dane (nie ma takiego rekordu w bazie danych bądź hasło nie pasuje do loginu). Po udanej operacji (logowanie powiodło się) należy stworzyć użytkownikowi sesję oraz przekierować go na stronę główną serwisu.
	
	\subsection{Backlog Item 6}
	\paragraph{Tytuł zadania:} Pasek nawigacji
	\paragraph{Opis zadania:} Utworzenie paska nawigacji umieszczonego w górnej części ekranu;
	\paragraph{Priorytet:} Wysoki
	\paragraph{Definition of Done:} Należy utworzyć pasek nawigacji, który będzie umiejscowiony na górze ekranu przeglądarki. Pozycjonowanie paska należy ustawić jako \emph{fixed}, aby w przypadku długiej strony ,,przykleił'' się do ekranu i podążał za scrollowaną stroną. Szerokość paska powinna być uzależniona od szerokości ekranu, na którym będzie odwiedzany serwis, zaś wysokość powinna być ustawiona na \emph{75px}.
	
	\subsection{Backlog Item 7}
	\paragraph{Tytuł zadania:} Zmiana danych
	\paragraph{Opis zadania:} Umożliwienie użytkownikowi zmiany swoich podstawowych danych;
	\paragraph{Priorytet:} Średni
	\paragraph{Definition of Done:} Każda osoba posiadająca swoje konto powinna mieć możliwość zmiany swoich podstawowych danych na nowe (tj. imię, nazwisko, e-mail, miejscowość, numer telefonu, numer konta bankowego, hasło). Należy stworzyć formularz z wcześniej wymienionymi danymi oraz jako domyślne \emph{value} kolejnych inputów formularza powinny być przypisane dane pobrane z bazy danych. Obowiązkowo należy zaimplementować przycisk \emph{submit}, po kliknięciu którego zostaną pobrane dane z formularza metodą \emph{POST} oraz wysłane do bazy danych w postaci zapytania MySQL. Wysłane dane muszą podmieniać już istniejące.
	
	\subsection{Backlog Item 8}
	\paragraph{Tytuł zadania:} Wyszukiwarka
	\paragraph{Opis zadania:} Umożliwienie użytkownikowi wyszukiwania interesujących przedmiotów;
	\paragraph{Priorytet:} Średni
	\paragraph{Definition of Done:} Każda osoba (zalogowana lub niezalogowana) może używać wyszukiwarki. Wyszukiwarke w postaci formularza należy umieścić na pasku nawigacji, który znajduje się na górze ekranu. Wpisane hasło w input formularza zwrócone przez funkcję \emph{htmlspecialchars()} należy umieścić w zapytaniu MySQL w poszukiwaniu przedmiotów. Zapytanie należy zaprojektować tak, aby wyszukiwał każdy przedmiot, w którym zawiera się wpisane przez użytkownika hasło \emph{(np. dom, domek, domeczek, domuś itp.)}.
	
	\subsection{Backlog Item 9}
	\paragraph{Tytuł zadania:} Panel konta
	\paragraph{Opis zadania:} Zgrupowanie w jedno miejsce podstawowych akcji użytkownika;
	\paragraph{Priorytet:} Średni
	\paragraph{Definition of Done:} Panel należy podzielić na dwie grupy - dla osób zalogowanych oraz dla osób niezalogowanych. Oba panele powinny być umiejscowione na pasku nawigacji na górze ekranu. Panel dla zalogowanego użytkownika powinien być w postaci avataru użytkownika, w który można kliknąć a następnie rozwija się menu z podstawowymi akcjami \emph{(tj. edytuj dane, moje aukcje, historia zakupów, wyloguj)}. Panel dla osoby niezalogowanej powinien być w postaci napisu ,,Moje konto'', które również rozwija się po kliknięciu w napis, z którego można wybrać dwie akcję \emph{(tj. zarejestruj się, zaloguj się)}. Po kliknięciu, w któryś z odnośników z rozwijanych menu każdy powinien być przekierowywany na konkretne podstrony serwisu.
	
	\subsection{Backlog Item 8}
	\paragraph{Tytuł zadania:} Dodanie aukcji
	\paragraph{Opis zadania:} Umożliwienie użytkownikowi wystawienia nowej aukcji;
	\paragraph{Priorytet:} Wysoki
	\paragraph{Definition of Done:} Każda osoba zalogowana na swoję konto powinna mieć tę opcję dostępną. Osoba, która nie jest zalogowana na swoje konto powinna dostać wiadomość zwrotną, że ta opcja jest tylko dla osób zalogowanych lub całkowicie zablokować dostęp. Tworzenie nowego rekordu w bazie danych w tabeli \emph{auction}.
	
	\subsection{Backlog Item 95}
	\paragraph{Tytuł zadania:} Formularz aukcyjny
	\paragraph{Opis zadania:} Umożliwienie użytkownikowi podanie niezbędnych danych dotyczących aukcji;
	\paragraph{Priorytet:} Wysoki
	\paragraph{Definition of Done:} Każda osoba podczas wystawiania nowego przedmiotu na akcję musi mieć możliwość wpisania niezbędnych danych, które pozwolą odróżniać od siebie inne aukcje (tj. konkretna nazwa aukcji, opis przedmiotu, cena, zdjęcie).
	
	
	\subsection{Backlog Item 10}
	\paragraph{Tytuł zadania:} Edytowanie aukcji
	\paragraph{Opis zadania:} Umożliwienie użytkownikowi edytowania danych swojej aukcji;
	\paragraph{Priorytet:} Średni
	\paragraph{Definition of Done:} Użytkownik, który wystawił przedmiot na aukcję powinien mieć możliwość edytowania danych związanych z aukcją. Zmienione dane powinny nadpisać się w bazie danych.
	
	\subsection{Backlog Item 11}
	\paragraph{Tytuł zadania:} Usuwanie aukcji
	\paragraph{Opis zadania:} Umożliwienie użytkownikowi usunięcia swojej aukcji;
	\paragraph{Priorytet:} Średni
	\paragraph{Definition of Done:} Użytkownik, który wystawił przedmiot na aukcję powinien mieć możliwość całkowitego usunięcia aukcji z serwisu. Usuwana aukcja powinna usunąć się również z bazy danych.
	
	
	\subsection{Backlog Item 12}
	\paragraph{Tytuł zadania:} Lista aukcji
	\paragraph{Opis zadania:} Umożliwienie użytkownikowi wyświetlania dostępnych aukcji;
	\paragraph{Priorytet:} Wysoki
	\paragraph{Definition of Done:} Każda osoba odwiedzająca serwis powinna mieć możliwość zobaczenia wszystkich dostępnych aukcji na stronie głównej.
	
	\subsection{Backlog Item 13}
	\paragraph{Tytuł zadania:} Kupowanie przedmiotu
	\paragraph{Opis zadania:} Umożliwienie użytkownikowi zakupu;
	\paragraph{Priorytet:} Średni
	\paragraph{Definition of Done:} Każda osoba posiadająca konto powinna mieć możliwość zakupu interesującego przedmiotu poprzez opcję KUP TERAZ lub wygranie licytacji.
	
	\subsection{Backlog Item 14}
	\paragraph{Tytuł zadania:} Informacja zwrotna
	\paragraph{Opis zadania:} Przesyłanie danych osób biorących udział w transakcji;
	\paragraph{Priorytet:} Wysoki
	\paragraph{Definition of Done:} Po sfinalizowaniu każdej transakcji powinny automatycznie przesyłać się dane:
	\begin{itemize}
		\item od kupującego do sprzedającego - imię, nazwisko, adres, numer telefonu;
		\item od sprzedającego do kupującego - imię, nazwisko, numer konta bankowego, numer transakcji, numer telefonu;
	\end{itemize}.
	
	\subsection{Backlog Item 15}
	\paragraph{Tytuł zadania:} Historia
	\paragraph{Opis zadania:} Umożliwienie użytkownikowi przeglądania swoich starych transakcji;
	\paragraph{Priorytet:} Niski
	\paragraph{Definition of Done:} Każda osoba posiadająca swoje konto powinna mieć możliwość przeglądania swoich wszystkich wcześniejszych transakcji w tym oznaczonych jako kupione, sprzedane lub wygasłe.
	
	
	\subsection{Backlog Item 16}
	\paragraph{Tytuł zadania:} Wiadomości
	\paragraph{Opis zadania:} Umożliwienie użytkownikowi wysyłania i odbierania prywatnych wiadomości od innych użytkowników;
	\paragraph{Priorytet:} Średni
	\paragraph{Definition of Done:} Każda osoba posiadająca konto powinna mieć możlwiość wysyłania prywatnych wiadomości do innych użytkowników serwisu oraz również odbierania.
	
	\subsection{Backlog Item 17}
	\paragraph{Tytuł zadania:} Komentarze
	\paragraph{Opis zadania:} Umożliwienie użytkownikowi komentowania konkretnych aukcji;
	\paragraph{Priorytet:} Niski
	\paragraph{Definition of Done:} Każda osoba zalogowana na swoje konto powinna mieć możliwość skorzystania z przycisku "Komentarz", który znajduje się przy przeglądanej aukcji i napisania swojego komentarza dot. ów aukcji w polu tekstowym, które powinno się pojawić po kliknięciu przycisku.
	
	
	\subsection{Backlog Item 18}
	\paragraph{Tytuł zadania:} Ulubione
	\paragraph{Opis zadania:} Umożliwienie użytkownikowi zapisania aukcji;
	\paragraph{Priorytet:} Niski
	\paragraph{Definition of Done:} Każda osoba zalogowana na swoje konto powinna mieć możliwość przypisania linku interesującej go aukcji do odpowiedniej zakładki w swoim profilu.
	
	\subsection{Backlog Item 19}
	\paragraph{Tytuł zadania:} Licznik
	\paragraph{Opis zadania:} Umożliwienie użytkownikowi sprawdzenia ilości odsłon aukcji;
	\paragraph{Priorytet:} Niski
	\paragraph{Definition of Done:} Na każdej stronie aukcji w dogodnym miejscu pod ogłoszeniem powinien znajdować się licznik (Wyświetleń:"ilość wyświetleń), który zapisze w sobie inf. ile razy ktoś wyświetlił ów ogłoszenie.
	



	

	
	\section{Sprint 1}
	\subsection{Cel} Celem pierwszego Sprintu jest umożliwienie użytkownikowi założenia konta dzięki któremu będzie on mógł uzyskać dostęp do wszystkich opcji serwisu i korzystania z niego. Aby to mogło się odbyć wcześniej zostanie stworzona baza danych w której będą zapisywane wszystkie dane.
	\subsection{Sprint Planning/Backlog}
	
	\paragraph{Tytuł zadania.} Baza danych.
	\begin{itemize}
		\item Estymata: XL
	\end{itemize}
	
	\paragraph{Tytuł zadania.} Rejestracja
	\begin{itemize}
		\item Estymata: S
	\end{itemize}
	
	\paragraph{Tytuł zadania.} Logowanie
	\begin{itemize}
		\item Estymata: S
	\end{itemize}
	
	\paragraph{<<Tutaj dodawać kolejne zadania>>}
	
	\subsection{Realizacja}
	
	\paragraph{Tytuł zadania.} <<Tytuł>>.
	\subparagraph{Wykonawca.} <<Wykonawca>>.
	\subparagraph{Realizacja.} <<Sprawozdanie z realizacji zadania (w tym ocena zgodności z estymatą). Kod programu (środowisko \texttt{verbatim}): \begin{verbatim}
	for (i=1; i<10; i++)
	...
	\end{verbatim}>>.
	
	\paragraph{Tytuł zadania.} <<Tytuł>>.
	\subparagraph{Wykonawca.} <<Wykonawca>>.
	\subparagraph{Realizacja.} <<Sprawozdanie z realizacji zadania (w tym ocena zgodności z estymatą). Kod programu (środowisko \texttt{verbatim}): \begin{verbatim}
	for (i=1; i<10; i++)
	...
	\end{verbatim}>>.
	
	\paragraph{<<Tutaj dodawać kolejne zadania>>}
	
	
	\subsection{Sprint Review/Demo}
	<<Sprawozdanie z przeglądu Sprint'u -- czy założony cel (przyrost) został osiągnięty oraz czy wszystkie zaplanowane Backlog Item'y zostały zrealizowane? Demostracja przyrostu produktu>>.
	
	\section{Sprint 2}
	
	\subsection{Cel} <<Określić, w jakim celu tworzony jest przyrost produktu>>.
	
	\subsection{Sprint Planning/Backlog}
	
	\paragraph{Tytuł zadania.} <<Tytuł>>.
	\begin{itemize}
		\item Estymata: <<szacowana czasochłonność (w ,,koszulkach'')>>.
	\end{itemize}
	
	\paragraph{Tytuł zadania.} <<Tytuł>>.
	\begin{itemize}
		\item Estymata: <<szacowana czasochłonność (w ,,koszulkach'')>>.
	\end{itemize}
	
	\paragraph{<<Tutaj dodawać kolejne zadania>>}
	
	\subsection{Realizacja}
	
	\paragraph{Tytuł zadania.} <<Tytuł>>.
	\subparagraph{Wykonawca.} <<Wykonawca>>.
	\subparagraph{Realizacja.} <<Sprawozdanie z realizacji zadania (w tym ocena zgodności z estymatą). Kod programu (środowisko \texttt{verbatim}): \begin{verbatim}
	for (i=1; i<10; i++)
	...
	\end{verbatim}>>.
	
	\paragraph{Tytuł zadania.} <<Tytuł>>.
	\subparagraph{Wykonawca.} <<Wykonawca>>.
	\subparagraph{Realizacja.} <<Sprawozdanie z realizacji zadania (w tym ocena zgodności z estymatą). Kod programu (środowisko \texttt{verbatim}): \begin{verbatim}
	for (i=1; i<10; i++)
	...
	\end{verbatim}>>.
	
	\paragraph{<<Tutaj dodawać kolejne zadania>>}
	
	
	\subsection{Sprint Review/Demo}
	<<Sprawozdanie z przeglądu Sprint'u -- czy założony cel (przyrost) został osiągnięty oraz czy wszystkie zaplanowane Backlog Item'y zostały zrealizowane? Demostracja przyrostu produktu>>.
	
	\section*{<<Tutaj dodawać kolejne Sprint'y>>}
	
	
	\begin{thebibliography}{9}
		
		\bibitem{Cov} S. R. Covey, {\em 7 nawyków skutecznego działania}, Rebis, Poznań, 2007.
		
		\bibitem{Oet} Tobias Oetiker i wsp., Nie za krótkie wprowadzenie do systemu \LaTeX  \ $2_\varepsilon$, \url{ftp://ftp.gust.org.pl/TeX/info/lshort/polish/lshort2e.pdf}
		
		\bibitem{SchSut} K. Schwaber, J. Sutherland, {\em Scrum Guide}, \url{http://www.scrumguides.org/}, 2016.
		
		\bibitem{apr} \url{https://agilepainrelief.com/notesfromatooluser/tag/scrum-by-example}
		
		\bibitem{us} \url{https://www.tutorialspoint.com/scrum/scrum_user_stories.htm}
		
	\end{thebibliography}
	
\end{document}
