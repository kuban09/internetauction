\documentclass[a4paper]{article}
\usepackage{polski}
\usepackage[cp1250]{inputenc}
\usepackage{url}
\usepackage{hyperref}

\title{\vspace{3cm} \bf{\Huge Internetowy\\ \vspace{0.2cm} Serwis\\ \vspace{0.4cm} Aukcyjny}}
\author{{\em Jakub Niewczas, Damian Klimek, Tomasz Zabrzewski, Łukasz Myszkowski}}
\date{}

\begin{document}
	
	\begin{titlepage}
		\thispagestyle{empty}
		\maketitle
		\vspace{8cm}
		\begin{center}
			Zespołowe przedsięwzięcie inżynierskie\\[2mm]
			
			Informatyka\\[2mm]
			
			Rok. akad. 2017/2018, sem. I\\[2mm]
			
			Prowadzący: dr hab. Marcin Mazur
		\end{center}
	\end{titlepage}
	
	\tableofcontents
	\thispagestyle{empty}
	
	\newpage
	
	\section{Opis projektu}
	
	\subsection{Członkowie zespołu}
	
	\begin{enumerate}
		\item Damian Klimek (kierownik projektu);
		\item Jakub Niewczas;
		\item Tomasz Zabrzewski;
		\item Łukasz Myszkowski;
	\end{enumerate}
	
	\subsection{Cel projektu (produkt)}
	
	Głównym i najważniejszym celem projektu jest stworzenie platformy webowej, która pojęcie ,,zakupy'' pchnie krok dalej. Potencjalni klienci będą mieć możliwość robienia zakupów \emph{(tj. sprzedawanie i kupowanie przedmiotów)} bez wychodzenia z domu. Kolejną ważną rzeczą jest zaimplementowanie modułu indywidualnych kont dla użytkowników \emph{(logowanie się na swoje konto oraz rejestrowanie nowego)}. Również stawiamy nacisk na stworzenie miłego dla oka, prostego a przede wszystkim intuicyjnego designu strony internetowej aby użytkownicy mogli w łatwy sposób przemieszczać się po platformie i szybko wyszukiwać interesujące ich przedmioty.
	
	\subsection{Potencjalny odbiorca produktu (klient)}
	
	Konkretnym klientem może być każdy, kto potrzebuje pieniędzy przez co dostaje możliwość sprzedania czegokolwiek lub osoby potrzebujące jakiegoś dobra. Widełki wieku klientów nie są definiowalne - każdy kto posiada dostęp do internetu może skorzystać z usług jakie oferuje produkt.
	
	\subsection{Metodyka}
	
	Projekt będzie realizowany przy użyciu (zaadaptowanej do istniejących warunków) metodyki {\em Scrum}. 
	
	\section{Wymagania użytkownika}
	
	\subsection{User story 1}
	Jako nowy użytkownik serwisu chciałbym mieć możliwość założenia swojego indywidualnego konta, żebym mógł korzystać w pełni z usług dostarczanych przez serwis.
	
	\subsection{User story 2}
	Jako użytkownik chciałbym mieć możliwość zalogować się na swoje indywidualne konto, by móc korzystać z wszystkich usług platformy.
	
	\subsection{User story 3}
	Jako użytkownik chciałbym mieć możliwość wystawienia przedmiotu na aukcję, żebym mógł sprzedawać przedmioty oraz zarabiać pieniądze.
	
	\subsection{User story 4}
	Jako użytkownik podczas wystawiania przedmiotu na aukcję chciałbym mieć możliwość wybrania formy aukcji \emph{(tj. licytacja, opcja ,,KUP TERAZ'' lub obie formy)}, żebym mógł sprzedać przedmiot w wybranej formie aukcji.
	
	\subsection{User story 5}
	Jako użytkownik podczas wystawiania przedmiotu na aukcję chciałbym mieć możliwość wprowadzenia opisu aukcji:
	\begin{itemize}
		\item konkretna nazwa aukcji;
		\item dodatkowe informację o sprzedawanym przedmiocie;
		\item cena \emph{(wywoławcza podczas licytacji lub stała za przedmiot podczas opcji ,,KUP TERAZ'')};
		\item wgranie zdjęcia przedmiotu;
	\end{itemize}
	żebym mógł zachęcić potencjalnego kupca do zakupu przedmiotu.
	
	\subsection{User story 6}
	Jako użytkownik chciałbym mieć możliwość komentowania produktów innych osób wystawiających przedmioty na aukcji, , żebym mógł dowiedzieć się interesujących informacji od innych osób, które wcześniej kupiły przedmiot na jego temat bądź podzielić się własną opinią. (Opcjonalnie).
	
	\subsection{User story 7}
	Jako użytkownik chciałbym mieć możliwość dodania przedmiotów które mnie interesują do zakładek \emph{ (tj. Ulubione)}, żebym mógł szybko i łatwo je odnaleźć. 
	
	\subsection{User story 8}
	Jako użytkownik chciałbym mieć możliwość dodania swoich danych:
	\begin{itemize}
		\item imię;
		\item nazwisko;
		\item miejscowość;
		\item numer telefonu;
		\item e-mail;
		\item numer konta bankowego;
	\end{itemize}
	żeby umożliwić innym użytkownikom kontakt ze mną i rozliczeń.
	
	\subsection{User story 9}
	Jako użytkownik chciałbym mieć możliwość zmiany:
	\begin{itemize}
		\item istniejącego  hasła na nowe;
		\item istniejącego e-maila na nowy;
		\item danych osobowe na nowe;
	\end{itemize}
	żeby zaktualizować email oraz wprowadzić nowe hasło bądź też poprawić błąd w danych osobowych lub całkowicie je zmienić.
	
	\subsection{User story 10}
	Jako użytkownik chciałbym mieć możliwość zobaczenia ile razy odwiedzono moją aukcję \emph{ (tj. licznik odsłon)}, żebym mógł zobaczyć jakie jest zainteresowanie moją aukcją.
	
	\subsection{User story 11}
	Jako użytkownik chciałbym mieć możliwość edytowania danych aukcji \emph{ (tj. cena, opis)} oraz usunięcia danej aukcji, żebym mógł poprawić błedy lub zaktualizować dane aukcji.
	
	\subsection{User story 12}
	Jako użytkownik chciałbym mieć możliwość wyszukiwania interesujących mnie aukcji poprzez podanie słów kluczowych w wyszukiwarce, żebym mógł znaleźć przedmiot, który chciałbym zakupić.
	
	\subsection{User story 13}
	Jako użytkownik chciałbym by po dokonaniu transakcji (kup teraz bądź wygrana licytacja) wysyłała się do kupującego wiadomość z moim numerem konta oraz informacjami dotyczącymi transakcji, by mógł on dokonać wpłaty za zakupiony przedmiot.
	
	\subsection{User story 14}
	Jako użytkownik chciałbym mieć możliwość sprawdzenia starych aukcji które zostały zakończone (przedmiot sprzedano, przedmiot kupiono) bądź też wygasły, żebym mógl sprawdzić historie swoich transakcji.
	 
	\subsection{User story 15}
	Jako użytkownik chciałbym mieć możliwość wysłania wiadomości prywatnej do innego użytkownika serwisu, żebym mógł się z nim skontaktować. 
	
	\subsection{User story 16}
	Jako użytkownik chciałbym mieć możliwość przeglądania wszystkich dostępnych aukcji na serwisie w postaci listy wraz z ich podstawowymi danymi \emph{(tj. nazwa aukcji, forma aukcji, cena, data rozpoczęcia oraz zakończenia aukcji)}, żebym mógł przeglądać aukcje oferowane przez serwis.
	
	\subsection{User story 17}
	Jako użytkownik chciałbym mieć możliwość zakupu produktu natychmiast (opcja kup teraz), bądź licytowania go (opcja licytuj), żebym mógł go kupić.
		
	\section{Harmonogram}
	
	\subsection{Rejestr zadań (Product Backlog)}
	
		\begin{itemize}
		\item Data rozpoczęcia: <<24.10.2017>>.
		\item  Data zakończenia: <<xx.xx.xxxx>>.
	\end{itemize}
	
	\subsection{Sprint 1}
	
	\begin{itemize}
		\item Data rozpoczęcia: \emph{31.10.2017};
		\item Data zakończenia: \emph{14.11.2017};
		\item Scrum Master: \emph{Łukasz Myszkowski};
		\item Product Owner: \emph{Tomasz Zabrzewski};
		\item Development Team: \emph{Jakub Niewczas, Damian Klimek}
	\end{itemize}
	
	\subsection{Sprint 2}
	
	\begin{itemize}
		\item Data rozpoczęcia: <<14.11.2017>>.
		\item  Data zakończenia: <<28.11.2017>>.
		\item Scrum Master: <<Tomasz Zabrzewski>>.
		\item Product Owner: <<Damian Klimek>>.
		\item Development Team: <<Łukasz Myszkowski, Jakub Niewczas>>.
	\end{itemize}

	\subsection{Sprint 3}

	\begin{itemize}
		\item Data rozpoczęcia: <<28.11.2017>>.
		\item  Data zakończenia: <<12.12.2017>>.
		\item Scrum Master: <<Damian Klimek>>.
		\item Product Owner: <<Jakub Niewczas>>.
		\item Development Team: <<Tomasz Zabrzewski, Łukasz Myszkowski>>.
	\end{itemize}
	
	\subsection{Sprint 4}

	\begin{itemize}
		\item Data rozpoczęcia: <<12.12.2017>>.
		\item  Data zakończenia: <<09.01.2018>>.
		\item Scrum Master: <<Jakub Niewczas>>.
		\item Product Owner: <<Damian Klimek>>.
		\item Development Team: <<Tomasz Zabrzewski, Łukasz Myszkowski>>.
	\end{itemize}

	\subsection{Sprint 5}

	\begin{itemize}
		\item Data rozpoczęcia: <<09.01.2018>>.
		\item  Data zakończenia: <<23.01.2018>>.
		\item Scrum Master: <<Łukasz Myszkowski>>.
		\item Product Owner: <<Tomasz Zabrzewski>>.
		\item Development Team: <<Damian Klimek, Jakub Niewczas>>.
	\end{itemize}

	
	\subsection*{<<Tutaj dodawać kolejne Sprint'y>>}
	
	\section{Product Backlog}
	
	% DODAĆ, że tworzym interfejs BAZA DANYCH itd.. w 1 Itemsie
	
	\subsection{Backlog Item 1}
	\paragraph{Tytuł zadania:} Baza danych
	\paragraph{Opis zadania:} Możliwość zapisania danych.
	\paragraph{Priorytet:} Wysoki
	\paragraph{Definition of Done:} Użytkownik po tym jak stworzy nowe konto będzie miał możliwość zapisania swoich danych w bazie. Wszystkie stworzone aukcje przez użytkownika będą również tam przechowywane.
	
	\subsection{Backlog Item 2}
	\paragraph{Tytuł zadania:} Logowanie
	\paragraph{Opis zadania:} Możliwość zalogowania się na swoje konto;
	\paragraph{Priorytet:} Wysoki
	\paragraph{Definition of Done:} Użytkownik posiadający konto na serwisie wpisując swoje dane \emph{(tj. login oraz hasło)}, może bez problemu zalogować się na nie. Należy zaimplementować wiadomość zwrotną gdy użytkownik wpisał złe dane.
	
	\subsection{Backlog Item 3}
	\paragraph{Tytuł zadania:} Rejestracja
	\paragraph{Opis zadania:} Możliwość założenia nowego konta;
	\paragraph{Priorytet:} Wysoki
	\paragraph{Definition of Done:} Każda osoba wchodząca na stronę serwisu musi posiadać możliwość założenia konta. Wpisując podstawowe dane konto musi zostać stworzone w bazie danych. Należy zaimplementować funckję sprawdzająca istniejące loginy użytkowników aby jedna nazwa użytkownika należała do jednej osoby.
	
	\subsection*{<<Tutaj dodawać kolejne zadania>>}
	
	\section{Sprint 1}
	\subsection{Cel} <<Określić, w jakim celu tworzony jest przyrost produktu>>.
	\subsection{Sprint Planning/Backlog}
	
	\paragraph{Tytuł zadania.} <<Tytuł>>.
	\begin{itemize}
		\item Estymata: <<szacowana czasochłonność (w ,,koszulkach'')>>.
	\end{itemize}
	
	\paragraph{Tytuł zadania.} <<Tytuł>>.
	\begin{itemize}
		\item Estymata: <<szacowana czasochłonność (w ,,koszulkach'')>>.
	\end{itemize}
	
	\paragraph{<<Tutaj dodawać kolejne zadania>>}
	
	\subsection{Realizacja}
	
	\paragraph{Tytuł zadania.} <<Tytuł>>.
	\subparagraph{Wykonawca.} <<Wykonawca>>.
	\subparagraph{Realizacja.} <<Sprawozdanie z realizacji zadania (w tym ocena zgodności z estymatą). Kod programu (środowisko \texttt{verbatim}): \begin{verbatim}
	for (i=1; i<10; i++)
	...
	\end{verbatim}>>.
	
	\paragraph{Tytuł zadania.} <<Tytuł>>.
	\subparagraph{Wykonawca.} <<Wykonawca>>.
	\subparagraph{Realizacja.} <<Sprawozdanie z realizacji zadania (w tym ocena zgodności z estymatą). Kod programu (środowisko \texttt{verbatim}): \begin{verbatim}
	for (i=1; i<10; i++)
	...
	\end{verbatim}>>.
	
	\paragraph{<<Tutaj dodawać kolejne zadania>>}
	
	
	\subsection{Sprint Review/Demo}
	<<Sprawozdanie z przeglądu Sprint'u -- czy założony cel (przyrost) został osiągnięty oraz czy wszystkie zaplanowane Backlog Item'y zostały zrealizowane? Demostracja przyrostu produktu>>.
	
	\section{Sprint 2}
	
	\subsection{Cel} <<Określić, w jakim celu tworzony jest przyrost produktu>>.
	
	\subsection{Sprint Planning/Backlog}
	
	\paragraph{Tytuł zadania.} <<Tytuł>>.
	\begin{itemize}
		\item Estymata: <<szacowana czasochłonność (w ,,koszulkach'')>>.
	\end{itemize}
	
	\paragraph{Tytuł zadania.} <<Tytuł>>.
	\begin{itemize}
		\item Estymata: <<szacowana czasochłonność (w ,,koszulkach'')>>.
	\end{itemize}
	
	\paragraph{<<Tutaj dodawać kolejne zadania>>}
	
	\subsection{Realizacja}
	
	\paragraph{Tytuł zadania.} <<Tytuł>>.
	\subparagraph{Wykonawca.} <<Wykonawca>>.
	\subparagraph{Realizacja.} <<Sprawozdanie z realizacji zadania (w tym ocena zgodności z estymatą). Kod programu (środowisko \texttt{verbatim}): \begin{verbatim}
	for (i=1; i<10; i++)
	...
	\end{verbatim}>>.
	
	\paragraph{Tytuł zadania.} <<Tytuł>>.
	\subparagraph{Wykonawca.} <<Wykonawca>>.
	\subparagraph{Realizacja.} <<Sprawozdanie z realizacji zadania (w tym ocena zgodności z estymatą). Kod programu (środowisko \texttt{verbatim}): \begin{verbatim}
	for (i=1; i<10; i++)
	...
	\end{verbatim}>>.
	
	\paragraph{<<Tutaj dodawać kolejne zadania>>}
	
	
	\subsection{Sprint Review/Demo}
	<<Sprawozdanie z przeglądu Sprint'u -- czy założony cel (przyrost) został osiągnięty oraz czy wszystkie zaplanowane Backlog Item'y zostały zrealizowane? Demostracja przyrostu produktu>>.
	
	\section*{<<Tutaj dodawać kolejne Sprint'y>>}
	
	
	\begin{thebibliography}{9}
		
		\bibitem{Cov} S. R. Covey, {\em 7 nawyków skutecznego działania}, Rebis, Poznań, 2007.
		
		\bibitem{Oet} Tobias Oetiker i wsp., Nie za krótkie wprowadzenie do systemu \LaTeX  \ $2_\varepsilon$, \url{ftp://ftp.gust.org.pl/TeX/info/lshort/polish/lshort2e.pdf}
		
		\bibitem{SchSut} K. Schwaber, J. Sutherland, {\em Scrum Guide}, \url{http://www.scrumguides.org/}, 2016.
		
		\bibitem{apr} \url{https://agilepainrelief.com/notesfromatooluser/tag/scrum-by-example}
		
		\bibitem{us} \url{https://www.tutorialspoint.com/scrum/scrum_user_stories.htm}
		
	\end{thebibliography}
	
\end{document}
